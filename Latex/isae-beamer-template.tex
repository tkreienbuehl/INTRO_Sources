% A LaTeX (non-official) template for ISAE projects reports
% Copyright (C) 2014 Damien Roque
% Version: 0.2
% Author: Damien Roque <damien.roque_AT_isae.fr>

\documentclass{beamer}
\usepackage[utf8]{inputenc}
\usepackage[frenchb]{babel}
\usepackage{palatino}
\usepackage{graphicx}
\graphicspath{{./images/}}
\usepackage{colortbl}
\usepackage{xcolor}
\usepackage{tikz}
\usetikzlibrary{shapes,arrows}
\usetikzlibrary{mindmap,trees}
\usetikzlibrary{calc}
\usepackage{pgfplots}
\pgfplotsset{compat=newest}
\pgfplotsset{plot coordinates/math parser=false}
\newlength\figureheight
\newlength\figurewidth
\usepackage{ifthen}
\usepackage{subfigure}
\usepackage{amsthm}
\usepackage{amsfonts}
\usepackage{amssymb}
\usepackage{amsmath}
\usepackage{eurosym}
\usepackage{wasysym}

% Printing on 2 slides per page
%\pgfpagesuselayout{2 on 1}[a4paper,border shrink=5mm]

% My macros...
\newcommand*{\SET}[1]  {\ensuremath{\boldsymbol{#1}}}
\newcommand*{\VEC}[1]  {\ensuremath{\boldsymbol{#1}}}
\newcommand*{\MAT}[1]  {\ensuremath{\boldsymbol{#1}}}
\newcommand*{\OP}[1]  {\ensuremath{\text{#1}}}
\newcommand*{\NORM}[1]  {\ensuremath{\left\|#1\right\|}}
\newcommand*{\DPR}[2]  {\ensuremath{\left \langle #1,#2 \right \rangle}}
\newcommand*{\calbf}[1]  {\ensuremath{\boldsymbol{\mathcal{#1}}}}
\newcommand*{\shift}[1]  {\ensuremath{\boldsymbol{#1}}}
\newcommand{\eqdef}{\stackrel{\mathrm{def}}{=}}
\newcommand{\argmax}{\operatornamewithlimits{argmax}}
\newcommand{\argmin}{\operatornamewithlimits{argmin}}
\newcommand{\ud}{\, \text{d}}
\newcommand{\vect}{\text{Vect}}
\newcommand{\sinc}{\text{sinc}}
\newcommand{\esp}{\ensuremath{\mathbb{E}}}
\newcommand{\hilbert}{\ensuremath{\mathcal{H}}}
\newcommand{\fourier}{\ensuremath{\mathcal{F}}}
\newcommand{\sgn}{\text{sgn}}
\newcommand{\intTT}{\int_{-T}^{T}}
\newcommand{\intT}{\int_{-\frac{T}{2}}^{\frac{T}{2}}}
\newcommand{\intinf}{\int_{-\infty}^{+\infty}}
\newcommand{\Sh}{\ensuremath{\boldsymbol{S}}}
\newcommand{\Cpx}{\ensuremath{\mathbb{C}}}
\newcommand{\R}{\ensuremath{\mathbb{R}}}
\newcommand{\Z}{\ensuremath{\mathbb{Z}}}
\newcommand{\N}{\ensuremath{\mathbb{N}}}
\newcommand{\K}{\ensuremath{\mathbb{K}}}
\newcommand{\reel}{\mathcal{R}}
\newcommand{\imag}{\mathcal{I}}
\newcommand{\cmnr}{c_{m,n}^\reel}
\newcommand{\cmni}{c_{m,n}^\imag}
\newcommand{\cnr}{c_{n}^\reel}
\newcommand{\cni}{c_{n}^\imag}
\newcommand{\LR}{\mathcal{L}_2(\R)}
\newcommand{\tproto}{g}
\newcommand{\rproto}{\check{g}}
\newcommand{\Tproto}{G}
\newcommand{\Rproto}{\check{G}}

%\theoremstyle{definition}
%\newtheorem{definition}{Définition}[subsection]

\theoremstyle{remark}
\newtheorem{remarque}{Remarque}[subsection]

\theoremstyle{plain}
\newtheorem{propriete}{Propriété}[subsection]
\newtheorem{exemple}{Exemple}[subsection]

% Choosing a main theme and a color theme
\mode<presentation> {
  %\usetheme{Warsaw}
  \usetheme{Madrid}
  %\usetheme{Frankfurt}
  \usecolortheme{seahorse}
}


\addtobeamertemplate{frametitle}{}{%
\vskip-1em
\begin{tikzpicture}[remember picture,overlay]
\node[anchor=north east,yshift=4pt] at (current page.north east) {\includegraphics[height=0.8cm]{images/logo-isae-long-sans-texte}};
\end{tikzpicture}}

\title[Titre court]{Titre long de la présentation}

\author[P. Nom et J. Doe]{\small Prénom Nom\inst{1} \and John Doe\inst{2}}

%\date{23 janvier 2014}

\institute[ISAE/DEOS]
{
\vspace{0.5cm}
\begin{minipage}{0.5\linewidth}
  \begin{center}
    \inst{1} ISAE Supaero, Département Electronique, Optronique, Signal (DEOS)\\
    \inst{2} ISAE Supaero, Département Mathématiques, Informatique, Automatique (DMIA)\\
    \vspace{1em}
    \includegraphics[height=2.5cm]{images/logo-isae-long}
  \end{center}
\end{minipage}
}

% Clear the navigation bar
\setbeamertemplate{navigation symbols}{}
 
\subject{Sujet de la présentation}

\begin{document}

\begin{frame}
\titlepage
\end{frame}

\begin{frame}
  \frametitle{Plan}
  \small
  \tableofcontents
  \normalsize
\end{frame}

% Recall the outline at each section
\AtBeginSection[]
{%
\begin{frame}
  \frametitle{Plan}
  \small
  %\tableofcontents[hideothersubsections]
  %\tableofcontents[currentsubsection,hideothersubsections]
  \tableofcontents[currentsubsection]
  \normalsize
\end{frame}
}

\section{Première partie}
\label{sec:partie1}

\begin{frame}
  \frametitle{Introduction}
  Voici un exemple.
  \begin{block}{Un exemple de bloc}
    Avec son contenu.
  \end{block}
  Une équation :
  \begin{equation}
    \label{eq:tranformee-fourier}
    S(f) = \intinf s(t) e^{-j2\pi f t} \ud t.
  \end{equation}
\end{frame}

\section{Deuxième partie}
\label{sec:partie2}

\begin{frame}[fragile]
  \frametitle{Un exemple avec deux colonnes}
  \begin{columns}
    \begin{column}{0.45\linewidth}
      Bla bla bla bla bla bla
      Une image
      \begin{center}
        \includegraphics[width=3cm]{images/logo-isae-court}
      \end{center}
    \end{column}
    \begin{column}{0.45\linewidth}
      Du code source
\begin{verbatim}
cd /home/user
make && make install
\end{verbatim}
      Une citation : \cite{Roque2012}.
    \end{column}
  \end{columns}
\end{frame}


\begin{frame}
  \frametitle{Questions}
  \begin{center}
    Merci pour votre attention.

    Avez-vous des questions ?
  \end{center}
\end{frame}

\newcounter{lastframe}
\setcounter{lastframe}{\insertframenumber}

\begin{frame}[allowframebreaks]{References}
\bibliographystyle{authoryear-fr}
\bibliography{references}
\end{frame}

\setcounter{framenumber}{\thelastframe}

\end{document}
